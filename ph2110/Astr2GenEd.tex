1.  In some of the astronomy lab exercises, direct measurement of lengths and
angles are made (similar to our the physics labs).  Other lab exercises provide
simulations of astronomical measurement that would be made with equipment
available to professionals.  In all of the exercises, the students collect
"data" of one kind or another and process the numbers to produce a result of
astronomical interest (spectra of atoms, distance and age of a star cluster,
etc.).

2. Most of the astronomy labs test the ideas and theories relevant to
astronomy, for example the laws of stellar evolution, and geometrical ideas of
parallax.

3. Both the lecture part of the course and the labs always deal with scientific
ideas and usually are set up to verify some sort of prediction.  In the lecture
part of the course, a lot of attention is paid to the difference between
scientific (predictive) and non-scientific (non-predictive) explanations of
nature.

4.  Unifying principles of astronomy include the formation of spectra by stars,
the principles of stellar evolution and using the laws of physics for finding
stellar distances. These are illustrated in the lab exercises.

I don't understand the meaning of "value of natural diversity". If it means
that a wide variety of topics in the field are discussed, that is certainly
true about the astronomy courses.

5. Impact of scientific discovery on human thought has great importance in the
sections of the course dealing with the history of astronomy.  Examples are the
size and age scales of the stars and the universe.
