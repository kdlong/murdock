\documentclass[12pt]{article}
\setlength{\textwidth}{6in}
\setlength{\textheight}{8.5in}
\setlength{\topmargin}{-0.25in}
\setlength{\oddsidemargin}{-.0in}
\setlength{\evensidemargin}{-.0in}
\title{\LaTeX\ }
\author{David Murdock, TTU}
\date{}

\begin{document}

\maketitle

\bigskip\bigskip

\section{Why You Need \LaTeX\ }

Eventually in your career as a physicist or mathematician your
ideas and findings will {\it matter\/} and you will need to
communicate those ideas to other people.  Computers and word
processors have made written communication relatively easy, but
most word processors do {\it not\/} meet then needs of physicists.
In physics and math we need to create documents which:

\smallskip
\noindent $\bullet$ Contain a lot of equations which we'd like to
have printed as clearly as possible.  ``Textbook quality'' is
desirable!

\smallskip
\noindent $\bullet$ Are formatted by the software, allowing us to
spend our time doing more physics and math. This includes the
numbering of the parts of a document, especially the equations,
figures and sections.

\smallskip
\noindent $\bullet$ Are {\it completely\/} transportable between
different computers and different operating systems on those
computers.  Scientists use different kinds of computers!

\smallskip
To meet these needs a computer scientist named Donald Knuth came
up with a powerful system of {\it typesetting\/} called \TeX.
Later, a programmer named Leslie Lamport made the system simpler
by changing the commands somewhat (this was possible because of
the re-programmable nature of the \TeX\ system) and created the
typesetting system known as \LaTeX.  \LaTeX\ has now become the
standard for producing high--quality documents in physics and
math.

It is true that much can be done with MS Word and its
``pull--down'' equation editor, and it is true that the ``learning
curve'' for the MS Word system is not as steep as with \LaTeX,
since modern word processors gives immediate visual confirmation
of what will show up on the paper. But the results are just not as
good.

\section{What Is \LaTeX?}

First off, \LaTeX\ is {\it not\/} a word processing program; it is
best called a {\it typesetting system\/}.  In an ordinary text
(``ASCII'') file you input the basic text of the document along
with special commands which will produce the desired formatting of
the document.  Included in the input are the equations, which have
their own special commands for positioning the symbols.

\section{What Do I Need To Use \LaTeX?}

\LaTeX\ can be used on any modern PC or Mac system. You will need
to get the software, of course, but there are {\it free\/}
versions to be downloaded from the web. You just need to get the
files and install them.

On many academic systems \LaTeX\ is already installed and someone
can tell you how to use it!

In addition to the \TeX\ software, you need a text editing
program.  All computer systems have these, but some editors are
more powerful than others.  UNIX and Windows systems come with
simple text editors but you can get better ones for free or at
small cost from the web.

\section{But Isn't It Complicated?}

It is true that if you want to use \LaTeX\ to write a book or make
a document with lots of illustrations then you will have to learn
lots of technical details about the formatting commands.  But you
can start simple, enter some text and a few equations and
gradually throw in more and more formatting tricks as you learn
them.

Some people put off learning about this powerful system because
they think they can't learn the entire system at once.  And they
{\it can't\/}\dots but there's no reason not to start simple and
work your way up!

\section{Then Show Me Something Simple!}

If you can create a text file with the following lines, you can
make a \LaTeX\ file!  Try:

\bigskip

\fbox{\fbox{
\parbox{3in}{\tt
 $\backslash$documentclass[12pt]$\{$article$\}$

 $\backslash$begin$\{$document$\}$

\vskip 12pt
 Hi Mom!

\vskip 12pt
 $\backslash$end$\{$document$\}$
}}}

\rm

\bigskip

If you can get this to work, you can move up to bigger things.

The file/document {\tt latextut.tex\/} included with this lecture
is intended to give you a brief sampling of the basic syntax of
\LaTeX\ but also to give you a running start in using the system.
You may want to grab this file and substitute your own text,
equations and tables for the silly material that is in that
document.

\section{Sounds Great, Dave. My Check Is in the Mail.
But What's the Catch?}

The catch is that \LaTeX\ is a system for creating your {\it
professional\/} documents; as such, it does not have some of the
convenient features of today's bloated word processors.
Specifically:

\medskip
Unless you are using a WYSIWYG front end for \LaTeX\, you
must {\it compile\/} the documents that you create, and there may
be errors.  You will have to fix them.

\medskip
In the basic installations of \LaTeX\ there are a limited number
of fonts you can use in a document.  This is not a limitation for
serious work because those cutesy fonts which have the i's dotted
with little hearts are usually out of place in {\sl Physical
Review}.

\medskip
Most importantly, it is not so easy for you to include
graphics with your documents.  You will find this to be a drawback
only at first; later on you can learn how to use a \LaTeX\
graphics ``package'' (for example the popular {\tt graphicx})
which will take the fine postscript figures that you have created
and put them in the document, properly sized.  Once you get the
hang of that, illustrations are not hard to put into a \LaTeX\
document.

\medskip

\section{So How Would I Make Up a Newsletter for My Club with
Pictures of Our Last Picnic and\dots}

Use MS Word.

\end{document}
