1.  In some of the astronomy lab exercises, direct measurement of lengths and
angles are made (similar to our the physics labs).  Other lab exercises provide
simulations of astronomical measurement that would be made with equipment
available to professionals.  In all of the exercises, the students collect
"data" of one kind or another and process the numbers to produce a result of
astronomical interest.  (Mass of a planet, distance and age of a star cluster,
etc.)

2. Most of the astronomy labs test the ideas and theories relevant to
astronomy, for example, Kepler's Laws, laws of stellar evolution, and
geometrical ideas of parallax.

3. Both the lecture part of the course and the labs always deal with scientific
ideas and usually are set up to verify some sort of prediction.  In the lecture
part of the course, a lot of attention is paid to the difference between
scientific and non-scientific explanations of nature, especially in the section
of the first semester where we cover the history of astronomy.

4.  Unifying principles of astronomy include the ideas of gravitation and using
the laws of physics for finding stellar distances.  These are illustrated in
the lab exercises.  In addition, the course includes the ideas of planetary
evolution.

I don't understand the meaning of "value of natural diversity".

5. Impact of scientific discovery on human thought has great importance in the
section of the course on the history of astronomy.  It also figures into the
second semester when we discuss discovery of the scale of the universe.
